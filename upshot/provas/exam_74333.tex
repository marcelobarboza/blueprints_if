\documentclass[addpoints,12pt]{exam}
\usepackage[utf8]{inputenc}
\usepackage[english,portuguese]{babel}
\usepackage[T1]{fontenc}
\usepackage{amsmath}
\usepackage{amssymb}
\usepackage{amsthm}
\hqword{Question}
\hpword{Points}
\hsword{Score}
\begin{document}
\begin{center}
    \begin{tabular*}{\textwidth}{l@{\extracolsep{\fill}}r}
        Instituto Federal Goiano & Campus Urutaí
    \end{tabular*}
\end{center}
\hrulefill
\begin{center}
    \begin{tabular*}{\textwidth}{l@{\extracolsep{\fill}}l@{\extracolsep{\fill}}r}
        Urutaí, 22 de Julho de 2020 & & Prof. Marcelo Barboza \\
        1\textsuperscript{\d a} Prova & Álgebra Moderna & Licenciatura em Matemática \\
        \texttt{Geovana Melo} & & \texttt{2017101221230060}
    \end{tabular*}
\end{center}
\hrulefill
\begin{center}
    \gradetable[h][questions]
\end{center}
\hrulefill
\begin{questions}
    \question[5]
    Seja $ G $ um grupo cíclico finito de ordem $ 935 $.
    Mostre que $ G $ possui um subgrupo de ordem $ d $, para cada
    $ d\in\{z\in\mathbb{Z}:0<d\mid 935 \} $.
    \question[5]
    Mostre que a função
    \[ \mathbb{R}\longrightarrow(0,\infty),\quad t\longmapsto 0.8089588006577774^t, \]
    é um isomorfismo de grupos.
\end{questions}
\end{document}
